
\documentclass{beamer}
\usetheme{Frankfurt}
\usecolortheme{default}
\usepackage{hyperref}
\usepackage[utf8]{inputenc} % 这是为了使用德语变音符号
\usepackage[english]{babel} % 这是为了使用德语变音符号
\usepackage[T1]{fontenc}    % 这是为了在pdf中正确输出变音符号

\begin{document}

\title{你的演示文稿标题}
\subtitle{副标题}
\author{Martin Thoma}
\date{2013年3月25日}
\subject{计算机科学}

\frame{\titlepage}

% 显示目录
\frame{
    \frametitle{目录}
    \setcounter{tocdepth}{1}
    \tableofcontents
    \setcounter{tocdepth}{2}
}

%\AtBeginSection[]{
%    \InsertToC[sections={\thesection}]  % 仅显示一个小节的子子节
%}

\section{介绍}
\subsection{一个小节!}
\begin{frame}{幻灯片标题}
    幻灯片内容
\end{frame}

\begin{frame}{另一个标题}
    一些文本\\
    \uncover<2->{在第2页幻灯片上显示我 (-)\\}
    \visible<3->{从第3页幻灯片开始可见 (-)\\}
    \only<4->{仅在第4页幻灯片上 (-)\\}
    \onslide<5->{在第5页及以后显示 (-)\\}
    \uncover<6>{在第6页幻灯片上显示我 \\}
    \visible<7>{在第7页可见\\}
    \only<8>{仅在第8页幻灯片上 \\}
    \alt<8>{我在第8页幻灯片上\\}{我不在第8页幻灯片上\\}
    \onslide<9>{在第9页幻灯片上\\}
\end{frame}

\begin{frame}{第三个标题}
    \begin{block}{标题}
        这是一个块。
    \end{block}

    \begin{exampleblock}{我的标题}
        这是一个示例块。
    \end{exampleblock}

    \begin{alertblock}{另一个标题}
        这是一个警告块。
    \end{alertblock}
\end{frame}

\end{document}
