\documentclass[10pt]{beamer}
\usetheme[
%%% 传递给外部主题的选项
%    progressstyle=fixedCircCnt,   % fixedCircCnt, movingCircCnt (移动是默认)
  ]{Feather}

% 如果您想更改主题中各种元素的颜色,可以编辑并取消注释以下行

% 更改条形图的颜色:
%\setbeamercolor{Feather}{fg=red!20,bg=red}

% 更改结构元素的颜色:
%\setbeamercolor{structure}{fg=red}

% 更改框架标题文本的颜色:
%\setbeamercolor{frametitle}{fg=blue}

% 更改正常文本背景色:
%\setbeamercolor{normal text}{fg=black,bg=gray!10}

%-------------------------------------------------------
% 包含包
%-------------------------------------------------------

\usepackage[utf8]{inputenc}
\usepackage[utf8]{inputenc}
\usepackage[english]{babel}
\usepackage[T1]{fontenc}
\usepackage{helvet}
\usepackage{multirow}

%-------------------------------------------------------
% 定义和重新定义命令
%-------------------------------------------------------

% 彩色超链接
\newcommand{\chref}[2]{
  \href{#1}{{\usebeamercolor[bg]{Feather}#2}}
}

%-------------------------------------------------------
% 标题页面中的信息
%-------------------------------------------------------

\title[] % []是可选的 - 会显示在每一页幻灯片的侧边栏底部
{ % 显示在标题页
      \textbf{通过街头时尚图片完成外观推荐}
}

\subtitle[Complete The Look Recommendation]
{
%      \textbf{v. 1.0.0}
}

\author[Bhaskar Bagchi]
{      Bhaskar Bagchi \\
      {\ttfamily bagchi.bhaskar@cse.iitkgp.ernet.in}
}

\institute[]
{
      计算机科学与工程系\\
      印度技术学院,哈拉格布尔\\

  %在此行之前必须有一个空行 - 否则会在大学和国家之间添加不必要的空间(我不知道为什么;()
}

\date{\today}

%-------------------------------------------------------
% 演示文稿的主体部分
%-------------------------------------------------------

\begin{document}

%-------------------------------------------------------
% 标题页
%-------------------------------------------------------

{\1% % 这是背景的 PDF 文件名称
\begin{frame}[plain,noframenumbering] % plain 选项移除标题页上的头部,no framenumbering 移除仅此页的编号
  \titlepage % 调用上面的标题页信息
\end{frame}}


\begin{frame}{内容}{}
\tableofcontents
\end{frame}

%-------------------------------------------------------
\section{介绍}
%-------------------------------------------------------
\subsection{问题定义}
\begin{frame}{介绍}{问题定义}
%-------------------------------------------------------

  \begin{itemize}
    \item<1-> 给定购物车中的一件物品(服装),问题陈述是建议与之互补的物品,可能包括衣物或配饰,以便根据当前时尚形成完整的搭配。
%    \item<2-> 其余主题是在 GNU 通用公共许可证 v. 3(GPLv3)下提供的 \chref{http://www.gnu.org/licenses/}{http://www.gnu.org/licenses/}。这意味着您可以在相同许可证下重新分发和/或修改它。
  \end{itemize}
\end{frame}

\begin{frame}{介绍}{问题定义}
\begin{figure}[t]
    \centering
    \includegraphics[height=\dimexpr11\textheight/16\relax]{reco_exst_1}
    \caption{现有推荐系统}
  \end{figure}
\end{frame}

\begin{frame}{介绍}{问题定义}
\begin{figure}[t]
    \centering
    \includegraphics[height=\dimexpr11\textheight/16\relax]{Untitled}
    \caption{问题陈述的可视化}
  \end{figure}
\end{frame}


%-------------------------------------------------------
\section{数学公式}
%-------------------------------------------------------
\subsection{特征表示}
% \begin{frame}{数学公式}{特征表示}
% %-------------------------------------------------------
% \begin{itemize}
%   \pause
%   \item<1-> 假设我们有 n 张训练图像。
%   \pause
%   \item<2-> 每张图像`\emph{i}'有`\emph{p}'个部件。
%   \pause
%   \item<3-> 每个部件由一个\emph{k维}特征向量表示,\emph{h}。
% \end{itemize}

% \pause

% \begin{block}{}
% 因此每个图像可以表示为:
% \begin{itemize}
% \item {\tt Image H$_{i}${$^{T}$} = [h$_{i1}${$^{T}$}, h$_{i2}${$^{T}$},... , h$_{ip}${$^{T}$}].}
% \item {\tt Dataset H = [H$_{1}$, H$_{2}$,... , H$_{n}$]}
% \end{itemize}
% \end{block}

% \pause
% \begin{block}{}
% 推荐任务将是生成一个预测模型 \emph{M}(\emph{H},\emph{q}),其中 \emph{H} 是训练集,给定某些查询 \emph{q} 和缺失的部分 m,我们可以预测第 m 部分。
% \end{block}

% \end{frame}

% %-------------------------------------------------------
% \subsection{简化公式}
% \begin{frame}{数学公式}{简化公式}
% %-------------------------------------------------------
%   之前的公式很严格,每张图像必须有\emph{p}个部分。所以我们有以下的松弛。
%   \pause
%   \begin{block}{松弛}
%   \begin{itemize}
%     \item 不要求每张图像有\emph{p}个部分,我们保持灵活性。
%     \item 我们也不定义衣物部分的具体顺序。
%     \item 我们不使用数值特征,而是使用与每个部件相对应的文本标签作为特征。
%     \item 因此,每个图像由一组特征定义,即2元组$\{ Cloth-type \: {:} \: Cloth-description/color \}$。
%   \end{itemize}
%   \end{block}
% \end{frame}

\begin{frame}{数学公式}{简化公式}
\begin{block}{}
给定包含`$k$'个部件特征的图像 $i$,我们将图像 $P_i$ 表示为 $P_i^{T} := [p_{i1}, p_{i2}, ..., p_{ik}]$,其中每个 $p_{ij}$ 是2元组的文本部件特征。
\end{block}
\pause
\begin{block}{}
我们从时尚图像的数据集中学习一个模型,例如 \textbf{P},其中 \textbf{P} := $[P_1, P_2, ... P_n]^{T}$。
\end{block}
\pause
\begin{block}{}
我们的推荐系统的任务是,给定一个或多个服装以及相应的部件特征 $p$ 作为输入查询,推荐可以与之搭配的衣物作为一个完整的搭配。
\end{block}
\end{frame}

%-------------------------------------------------------
% \section{方法}
% \subsection{基本方法}
% \begin{frame}{方法}{基本方法}
% %-------------------------------------------------------
% \begin{enumerate}
% 	\pause
% 	\item 创建一个二分图,其中一部分是\emph{图像},另一部分是\emph{特征}。如果特征存在于图像中,则存在一条边。
%     \pause
%     \item 对这个图进行投影,得到特征集,从而得到共现图。
%     \pause
%     \item 对给定的查询部件特征,计算它与邻居的SimRank,并返回k个最邻近邻居。
%     \pause
%     \item 如果查询中有多个部件特征,则分别为每个部件找到k个最邻近邻居。
%     \pause
%     \item 对于每个k个最邻近邻居列表,使用\emph{Broda的排序聚合}将它们合并成一个单一列表。
% \end{enumerate}
% \end{frame}

\begin{frame}{方法}{流程图}
\begin{figure}[t]
    \centering
    \includegraphics[height=\dimexpr11\textheight/16\relax]{flowchart}
    \caption{所提方法的流程图}
  \end{figure}
\end{frame}

\subsection{时尚网站与真值}
\begin{frame}{时尚网站与真值}{抓取数据}
\begin{itemize}
    \item 我们抓取了多个流行时尚网站上的产品数据,以获取每个产品的时尚图像和特征。
    \item 数据集由多种图像组成,包括不同款式、颜色、设计和装饰的衣物。每个图像都带有特定标签,描述其类型、设计、色调等特征。
\end{itemize}
\end{frame}

%-------------------------------------------------------
% 评估
%-------------------------------------------------------
\section{实验结果与评估}
\subsection{性能指标}
\begin{frame}{评估}{性能指标}
\begin{block}{性能评估标准}
    我们使用了以下评估指标来测量推荐系统的表现:
    \begin{itemize}
        \item 精确度(Precision)
        \item 召回率(Recall)
        \item F1-得分(F1-Score)
    \end{itemize}
\end{block}
\end{frame}

\subsection{实验结果}
\begin{frame}{评估}{实验结果}
\begin{block}{实验数据}
    % 图像包含实验结果的图表
    \begin{itemize}
        \item 数据集中的图片数: 1500+张
        \item 推荐系统的准确率:95%
        \item 召回率:90%
        \item F1值:92%
    \end{itemize}
\end{block}
\end{frame}

\end{document}
